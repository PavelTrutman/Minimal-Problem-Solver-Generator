\chapter{Conclusion}
In this work, we have focused on how to solve systems of polynomial equations fast and how to automatically generate efficient solvers for these systems.

In the first part, we have reviewed the state of the art algorithms for computing Gr\"obner basis of polynomial systems. We have described the Buchberger Algorithm \cite{Buchberger65}, then we have explained the $F_4$ Algorithm \cite{F4} in details and in the end, we have pointed out the main features of the $F_5$ Algorithm \cite{F5}.

In the second part, the automatic generator \cite{AutoGen} of minimal problem solvers has been presented. This tool enables us to easily generate solvers of systems of polynomial equations which arises when solving minimal problems in computer vision. We have described the process of generation of solvers in details. Then, we have suggested some improvements of the automatic generator and we have implemented them. For example, we have presented an improvement which allows us to generate multiple elimination solvers which are usually better for systems of polynomial equations in many unknowns. We have also shown that the solvers can be sped up when Gauss-Jordan eliminations for sparse matrices are used. Next, we have taken over a strategy from the $F_4$ Algorithm \cite{F4} and we have implemented it into the automatic generator. For better understanding, we have implemented the $F_4$ Algorithm \cite{F4} in Maple first. The description of this implementation is provided in this section, too. In the end, we have presented the benchmark of the automatic generator. This tool helps us to decide which generated solver is better for our application.

In the end, we have taken the 9-point relative pose different radial distortion problem \cite{9pt} and compared solvers generated with different methods for this problem on set of randomly generated data. We have shown that solvers generated with the new implemented methods may be faster than solvers generated by the old implementation of the automatic generator. We have register the most visible speed up when the $F_4$ strategy is used. In this case, the solver using the $F_4$ strategy is four times faster that the solver generated by the systematical generator for the 9-point relative pose problem \cite{9pt}.
