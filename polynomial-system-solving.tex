\chapter{Polynomial system solving}
Firstly we review the state of the art algorithms for computing Gr\"obner basis. Better understanding of these algorithms helps us to integrate them into polynomial solving algorithms based on Gr\"obner basis computation more efficiently.

\section{Buchberger Algorithm}
Buchberger Algorithm \cite{Buchberger65}, which was invented by Bruno Buchberger, was the first algorithm for computing Gr\"obner basis. The algorithm is described in details in \cite{Becker93, Cox-Little-Shea97}.

\subsection{First implementation}
The first and easy, but very inefficient implementation of the Buchberger Algorithm, Algorithm \ref{alg:simpleBuchberger}, is based on the observation that we can extend a set $F$ of polynomials to a Gr\"obner basis only by adding all non-zero remainders $\overline{S(f_i, f_j)}^F$ of all pairs from $F$ into $F$ until there is no non-zero remainder generated.

The main disadvantage of this simple algorithm is that so constructed Gr\"obner basis are often bigger than necessary. This implementation of the algorithm is also very inefficient because many of the S-polynomials that are constructed from the critical pairs are reduced to zero so after spending effort on computing them, there is nothing to add to the Gr\"obner basis $G$. How to decide which pairs need not be generated is described next.

\begin{algorithm}[ht]
  \begin{algorithmic}[1]
    \Require
      \Statex $F$ a finite set of polynomials
    \Ensure
      \Statex $G$ a finite set of polynomials
      \Statex

    \State $G \gets F$
    \State $B \gets \left\{\left\{g_1, g_2\right\}\ |\ g_1, g_2 \in G,\ g_1 \neq g_2\right\}$
    \While{$B \neq \emptyset$}
      \State select $\left\{g_1, g_2\right\}$ from $B$
      \State $B \gets B\backslash\left\{\left\{g_1, g_2\right\}\right\}$
      \State $h \gets S(g_1, g_2)$
      \State $h_0 \gets \overline{h}^G$
      \If{$h_0 \neq 0$}
        \State $B \gets B \cup \left\{\left\{g, h_0\right\}\ |\ g\in G\right\}$
        \State $G \gets G \cup \left\{h_0\right\}$
      \EndIf
    \EndWhile

    \State \Return $G$

  \end{algorithmic}
  \caption{Simple Buchberger Algorithm}
  \label{alg:simpleBuchberger}
\end{algorithm}



\subsection{Improved Buchberger Algorithm}
\label{subsec:ImprovedBuchberger}
The combinatorial complexity of the simple implementation of the Buchberger Algorithm can be reduced by testing out certain S-polynomials which need not be considered. To know which pairs can be deleted without treatment, we use the first and the second Buchberger's criterion \cite{Becker93}. Sometimes, we can even delete certain polynomials from the set $G$ completely, knowing that every critical pair they will generate will reduce to zero and hence these polynomials themselves will be superfluous in the output set. In the next few paragraphs we will describe the implementation of the Improved Buchberger Algorithm and of the function \textit{Update}, which deletes superfluous polynomials from $G$, according to Gebauer and M\"oller \cite{Gebauer-Moller88}.

The Improved Buchberger Algorithm, Algorithm \ref{alg:improvedBuchberger}, has the same structure as the Simple Algorithm. The function \textit{Update} is used at the beginning of the Improved Buchberger Algorithm to initialize the set $B$ of critical pairs and the Gr\"obner basis $G$ from the input set $F$ of polynomials and at every moment when a new non-zero polynomial $h_0 = \overline{h}^G$ of an S-polynomial $h$ has been found and the sets $B$ and $G$ are about to be updated.

\begin{algorithm}[ht]
  \begin{algorithmic}[1]
    \Require
      \Statex $F$ a finite set of polynomials
    \Ensure
      \Statex $G$ a finite set of polynomials
      \Statex

    \State $G \gets \emptyset$
    \State $B \gets \emptyset$
    \While{$F \neq \emptyset$}
      \State select $f$ from $F$
      \State $F \gets F\backslash\{f\}$
      \State $(G, B) \gets Update(G, B, f)$
    \EndWhile
    \While{$B \neq \emptyset$}
      \State select $\{g_1, g_2\}$ from $B$
      \State $B \gets B\backslash \left\{\left\{g_1, g_2\right\}\right\}$
      \State $h \gets S(g_1, g_2)$
      \State $h_0 \gets \overline{h}^G$
      \If{$h_0 \neq 0$}
        \State $(G, B) \gets Update(G, B, h_0)$
      \EndIf
    \EndWhile

    \State \Return $G$

  \end{algorithmic}
  \caption{Improved Buchberger Algorithm}
\end{algorithm}


Now, let us look at the function \textit{Update}, Algorithm \ref{alg:update}. First, it makes pairs from the new polynomial $h$ and all polynomials from the set $G_{old}$ and puts them into the set $C$. The first while loop (lines \ref{alg:update:w1b} -- \ref{alg:update:w1e}) iterates over all pairs in the set $C$. In each iteration it select a pair $\{h, g_1\}$ from the set $C$ and removes it from the set. Then it looks for another pair $\{h, g_2\}$ from the set $C$ or the set $D$. If does not exists a pair $\{h, g_2\}$ such that $(h, g_2, g_1)$ is a Buchberger triple, then the pair $\{h, g_1\}$ is put into the set $D$. The triple $(h, g_2, g_1)$ of polynomials $h$, $g_1$ and $g_2$ is a Buchberger triple if the equivalent conditions 

\begin{eqnarray}
	\LM(g_2) &|& \lcm(\LM(h), \LM(g_1))\\
	\lcm(\LM(h), \LM(g_2) ) &|& \lcm(\LM(h), \LM(g_1))\\
	\lcm(\LM(g_2), \LM(g_1)) &|& \lcm(\LM(h), \LM(g_1))
\end{eqnarray}
are satisfied. From the second Buchberger's criterion, we know that if a Buchberger triple $(h, g_2, g_1)$ shows up in the Buchberger Algorithm and the pairs $\{g_1, g_2\}$ and $\{h, g_2\}$ are amongs the critical pairs, then the pair $\{h, g_1\}$ need not be generated. That means in the code that such a pair is not moved from the set $C$ to the set $D$ but it is only removed from the set $C$. This while loop keeps all pairs $\{h, g_1\}$ where LM$(h)$ and LM$(g_1)$ are disjoint, i.e. LM$(h)$ and LM$(g_1)$ have no variable in common. The reason of this is that if two or more pairs in $C$ have the same lcm of their leading monomials, then there is a choice which one should be deleted. So we keep the pair where the leading monomials are disjoint. Pairs with disjoint leading monomials are removed in the second while loop, so we eventually remove them all.

The second while loop (lines \ref{alg:update:w2b} -- \ref{alg:update:w2e}) eliminates all pairs with disjoint leading monomials. We can remove such pairs thanks to the first Buchberger's criterion. All remaining pairs are stored in the set $E$.

The third while loop (lines \ref{alg:update:w3b} -- \ref{alg:update:w3e}) eliminates pairs $\{g_1, g_2\}$ where $(g_1, h, g_2)$ is a Buchberger triple from the set $B_{old}$. Then the updated set of the old pairs and the new pairs are united into the set $B_{new}$.

Finally, the last while loop (lines \ref{alg:update:w4b} -- \ref{alg:update:w4e}) removes all polynomials $g$ whose leading monomial is a multiple of the leading monomial of $h$ from the set $G_{old}$. We can eliminate such polynomials for two reasons. Firstly, LM$(h) \mid$ LM$(g)$ implies LM$(h) \mid$ lcm(LM$(g)$, LM$(f)$) for arbitrary polynomial $f$. We can see that $(g, h, f)$ is a Buchberger triple for any $f$ which in future appears in the set $G$. Moreover, polynomial $g$ will not be missed at the end, because in the Gr\"obner basis $G$, polynomials with leading monomials which are multiples of leading monomials of another polynomial from $G$ are superfluous, i.e. they will be eliminated in the reduced Gr\"obner basis.

In the end of the function, the polynomial $h$ is added into the Gr\"obner basis $G_{new}$. The output of the function \textit{Update} is the Gr\"obner basis $G_{new}$ and the set $B_{new}$ of critical pairs.

\begin{algorithm}[htp]
  \begin{algorithmic}[1]
    \Require
      \Statex $G_{old}$ a finite set of polynomials
      \Statex $B_{old}$ a finite set of pairs of polynomials
      \Statex $h$ a polynomial such that $h \neq 0$
    \Ensure
      \Statex $G_{new}$ a finite set of polynomials
      \Statex $B_{new}$ a finite set of pairs of polynomials
      \Statex

    \State $C \gets \left\{\left\{h, g\right\}\ |\ g\in G_{old}\right\}$
    \State $D \gets \emptyset$
    \While{$C\neq \emptyset$}
      \State select $\{h,g_1\}$ from $C$
      \State $C \gets C\backslash \{\{h, g_1\}\}$
      \IfML{LM$(h)$ and LM$(g_1)$ are disjoint \textbf{or}}
        \StatexIndent[3](lcm(LM$(h)$, LM$(g_2)$) $\nmid$ lcm(LM$(h)$, LM$(g_1)$) for all $\{h, g_2\}\in C$ \textbf{and}
        \StatexIndent[3]lcm(LM$(h)$, LM$(g_2)$) $\nmid$ lcm(LM$(h)$, LM$(g_1)$) for all $\{h, g_2\} \in D$) \algorithmicthen
        \State $D \gets D \cup \{\{h, g_1\}\}$
      \EndIf
    \EndWhile

    \State $E \gets \emptyset$

    \While{$D \neq \emptyset$}
      \State select $\{h, g\}$ from $D$
      \State $D \gets D\backslash \{\{h, g\}\}$
      \If{LM$(h)$ and LM$(g)$ are not disjoint}
        \State $E \gets E \cup \{\{h, g\}\}$
      \EndIf
    \EndWhile

    \State $B_{new} \gets \emptyset$

    \While{$B_{old} \neq \emptyset$}
      \State select $\{g_1, g_2\}$ from $B_{old}$
      \State $B_{old} \gets B_{old} \backslash \{\{g_1, g_2\}\}$
      \IfML{LM$(h) \nmid$ lcm(LM$(g_1)$, LM$(g_2)$) \textbf{or}}
        \StatexIndent[3]lcm(LM$(g_1)$, LM$(h)) = $ lcm(LM$(g_1)$, LM$(g_2)$) \textbf{or}
        \StatexIndent[3]lcm(LM$(h)$, LM$(g_2)) = $ lcm(LM$(g_1)$, LM$(g_2)$) \algorithmicthen
        \State $B_{new} \gets B_{new} \cup \{\{g_1, g_2\}\}$
      \EndIf
    \EndWhile

    \State $B_{new} \gets B_{new} \cup E$
    \State $G_{new} \gets \emptyset$
    
    \While{$G_{old} \neq \emptyset$}
      \State select $g$ from $G_{old}$
      \State $G_{old} \gets G_{old} \backslash \{g\}$
      \If{LM$(h)$ $\nmid$ LM$(g)$}
        \State $G_{new} \gets G_{new} \cup \{g\}$
      \EndIf
    \EndWhile

    \State \Return $(G_{new}, B_{new})$

  \end{algorithmic}
  \caption{Update}
\end{algorithm}


\section{$F_4$ Algorithm}
The $F_4$ Algorithm \cite{F4} by Jean-Charles Faug\`ere is an improved version of the Buchberger's Algorithm. The $F_4$ replaces the classical polynomial reduction found in the Buchberger's Algorithm by a simultaneous reduction of several polynomials. This reduction mechanism is achieved by a symbolic precomputation followed by Gaussian elimination implemented using sparse linear algebra methods. $F_4$ speeds up the reduction step by exchanging multiple polynomial divisions for row-reduction of a single matrix.

\subsection{Improved Algorithm $F_4$}
The main function of the $F_4$ Algorithm, Algorithm \ref{alg:F4}, consists of two parts. The goal of the first part is to initialize the whole algorithm.

First, it generates the set $P$ of critical pairs and initializes the Gr\"obner basis $G$. This is done by taking each polynomial from the input set $F$ and calling the function \textit{Update} on it, which updates the set $P$ of pairs and the set $G$ of basic polynomials.

The second part of the algorithm generates new polynomials and adds them into the set $G$. In each iteration, it selects some pairs from $P$ using the function \textit{Sel}. Many selection strategies are possible and is still an open question how to best select the pairs. Some selection strategies are described in the section \ref{subsec:F4:sel} on page \pageref{subsec:F4:sel}. Then, it splits each selected pair $\{f_1, f_2\}$ into two tuples. The first tuple contains the first polynomial $f_1$ of the pair and the monomial $m_1$ such that $\textrm{LM}(m_1 \times f_1) = \textrm{lcm}(\textrm{LM}(f_1),\textrm{LM}(f_2))$. The second tuple is constructed in the same way from the second polynomial $f_2$ of the pair. All tuples from all selected pairs are put into the set $L$, i.e. duplicates are removed.

Next, function \textit{Reduction} is called on the set $L$. It stores result in the set $\tilde{F}^+$. In the end of the algorithm it iterates through all new polynomials in the set $\tilde{F}^+$ and calls the function \textit{Update} on each of them. This generates new pairs into the set $P$ of critical pairs and extends the Gr\"obner basis $G$.

This algorithm terminates when the set $P$ of pairs is empty. Then the set $G$ is a Gr\"obner basis and it is the output of the algorithm.

\begin{algorithm}[ht]
  \begin{algorithmic}[1]
    \Require
      \Statex $F$ a finite set of polynomials
      \Statex $Sel$ a function $List(Pairs) \to List(Pairs)$ such that $Sel(l) \neq \emptyset$ if $l\neq\emptyset$
    \Ensure
      \Statex $G$ a finite set of polynomials
      \Statex

    \State $G \gets \emptyset$
    \State $P \gets \emptyset$
    \State $d \gets 0$

    \While{$F \neq \emptyset$}
      \State select $f$ form $F$
      \State $F \gets F\backslash \{f\}$
      \State $(G, P) \gets Update(G, P, f)$
    \EndWhile
    
    \While{$P\neq\emptyset$}
      \State $d \gets d + 1$
      \State $P_d \gets Sel(P)$
      \State $P \gets P\backslash P_d$
      \State $L_d \gets Left(P_d) \cup Right(P_d)$
      \State $(\tilde{F}^+_d, F_d) \gets Reduction(L_d, G, (F_i)_{i=1,\ldots,(d-1)})$
      \For{$h\in\tilde{F}^+_d$}
        \State $(G, P) \gets Update(G, P, h)$
      \EndFor
    \EndWhile
    \State \Return $G$

  \end{algorithmic}
  \caption{Improved Algorithm $F_4$}
  \label{alg:F4}
\end{algorithm}



\subsection{Function Update}
In the $F_4$ Algorithm the standard implementation of the Buchberger's Criteria such as the Gebauer and M\"oller installation \cite{Gebauer-Moller88} is used. Details about the function \textit{Update} can be found in the section \ref{subsec:ImprovedBuchberger}. The pseudocode of the function is shown in Algorithm \ref{alg:update}.

\subsection{Function Reduction}
Function \textit{Reduction}, Algorithm \ref{alg:reduction}, performs polynomial division using methods of linear algebra.

Input of the function \textit{Reduction} is a set $L$ containing tuples of monomial and polynomial. These tuples were constructed in the main function of the $F_4$ Algorithm from all selected pairs.

First, the function \textit{Reduction} calls the function \textit{Symbolic Preprocessing} on the set $L$. This returns a set $F$ of polynomials to be reduced. To use linear algebra methods to perform polynomial division, the polynomials have to be represented by a matrix. Each column of the matrix corresponds to a monomial. Columns have to be ordered with respect to the monomial ordering used so that the most right column corresponds to ``1''. Each row of the matrix corresponds to a polynomial from the set $F$. The matrix is constructed as follows. On the $(i, j)$ position in the matrix, we put the coefficient of the term corresponding to $j$-th monomial from the $i$-th polynomial from the set $F$.

We next reduce the matrix to a row echelon form using, for example, Gauss-Jordan elimination. Note that this matrix is typically sparse so we can use sparse linear algebra methods to save computation time and memory. After elimination, we construct resulting polynomials by multiplying the reduced matrix by a vector of monomials from the right.

In the end, the function returns the set $\tilde{F}^+$ of reduced polynomials such that their leading monomials are not leading monomials of any polynomial from the set $F$ of polynomials before reduction.

\begin{algorithm}[ht]
  \begin{algorithmic}[1]
    \Require
      \Statex $L$ a finite set of tuples of monomial and polynomial
      \Statex $G$ a finite set of polynomials
      \Statex $\mathcal{F} = (F_i)_{i=1,\ldots,(d-1)}$, where $F_i$ is finite set of polynomials
    \Ensure
      \Statex $\tilde{F}^+$ a finite set of polynomials
      \Statex $F$ a finite set of polynomials
      \Statex

    \State $F \gets Symbolic\ Preprocessing(L, G, \mathcal{F})$
    \State $\tilde{F} \gets$ Reduction to a Row Echelon Form of $F$
    \State $\tilde{F}^+ \gets \left\{f \in \tilde{F}\ |\ \textrm{LM}(f) \notin \textrm{LM}(F)\right\}$

    \State \Return $(\tilde{F}^+, F)$

  \end{algorithmic}
  \caption{Reduction}
\end{algorithm}



\subsection{Function Symbolic Preprocessing}
Function \textit{Symbolic Preprocessing}, Algorithm \ref{alg:symbolicPreprocessing}, starts with a set $L$ of tuples each containing a monomial and a polynomial. These tuples were constructed in the main function of the $F_4$ Algorithm from the selected pairs. Then, the tuples are simplified by the function \textit{Simplify} and after multiplying polynomials with corresponding monomials, the results are put into the set $F$.

Next, the function goes through all monomials in the set $F$ and for each monomial $m$ looks for some polynomial $f$ from the Gr\"obner basis $G$ such $m = m^\prime \times \textrm{LM}(f)$ where $m^\prime$ is some monomial. All such polynomials $f$ and monomials $m^\prime$ are after simplification multiplied and put into the set $F$. The goal of this search is to have for every monomial in $F$ a polynomial in $F$ with the same leading monomial. This will ensure that all polynomials from $F$ will be reduced for $G$ after polynomial division by linear algebra.

\begin{algorithm}[ht]
  \begin{algorithmic}[1]
    \Require
      \Statex $L$ a finite set of tuples of monomial and polynomial
      \Statex $G$ a finite set of polynomials
      \Statex $\mathcal{F} = (F_i)_{i=1,\ldots,(d-1)}$, where $F_i$ is finite set of polynomials
    \Ensure
      \Statex $F$ a finite set of polynomials
      \Statex

    \State $F \gets \big\{multiply($\textit{Simplify}$(m, f, \mathcal{F}))\ |\ (m, f)\in L\big\}$
    \State $Done \gets \textrm{LM}(F)$
    \While{$\textrm{M}(F) \neq Done$}
      \State $m$ an element of $\textrm{M}(F)\backslash Done$
      \State $Done \gets Done \cup \{m\}$
      \If{$m$ is top reducible modulo $G$}
        \State $m = m^\prime \cdot\textrm{LM}(f)$ for some $f \in G$ and some monomial $m^\prime$
        \State $F \gets F \cup \big\{multiply($\textit{Simplify}$(m^\prime, f, \mathcal{F}))\big\}$
      \EndIf
    \EndWhile

    \State \Return $F$

  \end{algorithmic}
  \caption{Symbolic Preprocessing}
  \label{alg:symbolicPreprocessing}
\end{algorithm}



\subsection{Function Simplify}
The function \textit{Simplify}, Algorithm \ref{alg:simplify}, simplifies a polynomial $m \times f$ which is a product of a given monomial $m$ and a polynomial $f$.

The function recursively looks for a monomial $m^\prime$ and a polynomial $f^\prime$ such that $\textrm{LM}(m^\prime\times f^\prime) = \textrm{LM}(m\times f)$. The polynomial $f^\prime$ is selected from all polynomials that have been reduced in previous iterations (sets $\tilde{F}$). We select polynomial $f^\prime$ such that the total degree of $m^\prime$ is minimal.

This is done in the function \textit{Symbolic Preprocessing} to insert polynomials that are mostly reduced and have a small number of monomials into the set $F$ of polynomials to be reduced. This of course speeds up following reduction.

\begin{algorithm}[ht]
  \begin{algorithmic}[1]
    \Require
      \Statex $m$ a monomial
      \Statex $f$ a polynomial
      \Statex $\mathcal{F} = (F_i)_{i=1,\ldots,(d-1)}$, where $F_i$ is finite set of polynomials
    \Ensure
      \Statex $(m^\prime, f^\prime)$ a non evaluated product of a monomial and a polynomial
      \Statex

    \For{$u \in$ list of all divisors of $m$}
      \If{$\exists j\ (1 \leq j \le d)$ such that $(u\times f) \in F_j$}
        \State $\tilde{F}_j$ is the Row Echelon Form of $F_j$
        \State there exists a (unique) $p\in \tilde{F}^+_j$ such that $\textrm{LM}(p) = \textrm{LM}(u\times f)$
        \If{$u\neq m$}
          \State \Return $Simplify(\frac{m}{u}, p, \mathcal{F})$
        \Else
          \State \Return $(1, p)$
        \EndIf
      \EndIf
    \EndFor

    \State \Return $(m, f)$

  \end{algorithmic}
  \caption{Simplify}
\end{algorithm}



\subsection{Selection strategy}
\label{subsec:F4:sel}
For the speed of the $F_4$ Algorithm, it is very important how the critical pairs from the list of all critical pairs $P$ are selected in each iteration. This of course depends on the implementation of the function \textit{Sel}. There are more possible selection strategies:

\begin{itemize}
  \item The easiest implementation is to select all pairs from $P$. In this case we reduce all critical pairs at the same time.
  \item If the function \textit{Sel} selects only one critical pair then the $F_4$ Algorithm is the Buchberger's Algorithm. In this case the \textit{Sel} function corresponds to the selection strategy in the Buchberger's Algorithm.
  \item The best function that Faug\`ere has tested is to select all critical pairs with a minimal total degree. Faug\`ere calls this strategy the \textit{normal strategy for} $F_4$. Pseudocode of this function can be found as Algorithm \ref{alg:sel}.
\end{itemize}

\begin{algorithm}[ht]
  \begin{algorithmic}[1]
    \Require
      \Statex $P$ a list of critical pairs
    \Ensure
      \Statex $P_d$ a list of critical pairs
      \Statex

    \State $d \gets min\left\{\textrm{deg}(\textrm{lcm}(p))\ |\ p\in P\right\}$
    \State $P_d \gets \left\{p\in P\ |\ \textrm{deg}(\textrm{lcm}(p)) = d\right\}$
    \State \Return $P_d$
  \end{algorithmic}
  \caption{Sel -- The normal strategy for $F_4$}
\end{algorithm}



\section{$F_5$ Algorithm}
