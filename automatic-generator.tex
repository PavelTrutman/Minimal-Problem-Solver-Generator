\chapter{Automatic generator}
The automatic generator of Gr\"obner basis solvers is used to easily solve problems leading to systems of polynomial equations. These systems usually arise when solving minimal problems \cite{MinimalProblems} in computer vision. Typically, these systems are not trivial so special solvers have to be designed for concrete problems to achieve efficient and numerically stable solvers. But solvers generated for concrete problems can not be easily applied for similar or new problems and therefore the automatic generator was proposed in \cite{AutoGen}. Solvers generated by the automatic generator can be easily used by non-experts users to solve complex problems.

The input of the automatic generator is a system of polynomial equations with a finite number of solutions and the output is a MATLAB or a Maple code that computes solutions of the given system for arbitary coefficients. One of the goals of this thesis is to improve previous implementation \cite{AutoGen} of the automatic generator to construct more efficient and numerically stable solvers.

\section{Description of the automatic generator}
	In this section we would like to briefly describe the procedure for generating solvers. The automatic generator consists of several independent modules, see Figure \ref{autogen:blockDiagram}. Since all these modules are independent, they can be easily improved or replaced by more efficient implementations. Next we describe each of these modules, full description can be found in \cite{AutoGen}.

\begin{figure}[ht]
  \centering
  \includegraphics[width=0.95\textwidth]{AutomaticGenerator.pdf}
  \caption{Block diagram of the automatic generator}
  \label{autogen:blockDiagram}
\end{figure}

\section{Reimplementation}

\section{Multiple eliminations solver}

\section{Removing unnecessary polynomials}

\section{Matrix partitioning}

\section{F4 strategy}
