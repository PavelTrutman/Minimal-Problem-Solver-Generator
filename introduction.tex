\chapter{Introduction}
\section{Motivation}
Many problems in computer vision can be formulated using systems of algebraic equations. Examples are the minimal problems \cite{MinimalProblems} which arise when computing geometrical models from image data. The polynomial systems arising from this problems are often not trivial and they consists of many polynomial equations of higher degree in~many unknowns, and therefore general algorithms for solving polynomial systems are not efficient for them. So special solvers for each problem have been developed to solve these system efficiently and numerically robustly.

Minimal problems have a wide range of applications, for example in 3D reconstruction, recognition, robotics and augmented reality. In these applications, the solvers of~minimal problems are only small parts of large computation systems which are supposed be fast or even to work in real-time applications. Moreover, these systems need to compute the solutions of the minimal problems repeatedly for large number of parameters. Therefore, very efficient solvers are required in computer vision.

Many solvers for minimal problems have been designed ad hoc for concrete problems, and therefore they can not be used or easily modified to solve different or even similar problems. So the automatic generator \cite{AutoGen} has been proposed. This tool generates Gr\"obner basis solvers automatically which enables us to generate an efficient solver for each problem we want to solve.

There are several ways, how the solver using Gr\"obner basis methods can be generated. The implementation presented in \cite{AutoGen} generates polynomials that are required for solving the system systematically. But other methods can be used. In this thesis, we review the state of the art methods for solving polynomials systems and suggest which methods can be taken over to improve the automatic generator and we implement them.

The automatic generator deals with sparse matrices in most cases. Therefore, we may consider to implement some methods which enable us to work with sparse matrices in an~efficient way to save computation time and memory. In this thesis, we focus on how to improve the Gauss-Jordan elimination of sparse matrices. We use the recent work \cite{SBBD} which presents the significant speed up of Gauss-Jordan elimination of sparse matrices. The speed up is caused by transforming matrices into the singly-bordered block-diagonal forms by the paritioning tool PaToH \cite{PaToH}. This method is based on the~fact that more eliminations of smaller matrices are faster than one elimination of a~big matrix.

\section{Thesis structure}
In this thesis, we firstly review the state of the art methods for computing Gr\"obner basis of polynomial systems. We start with describing simple, but easily understandable, algorithms and continue with more difficult, but also more efficient, algorithms. It is crucial to us to better understand these algorithms because we will use some techniques from them to improve the automatic generator lately in this thesis.

Secondly, we briefly describe the automatic generator. Then, we suggest some improvements of the automatic generator to generate efficient and numerically stable solvers. Some techniques implementened in the automatic generator may be efficient for some minimal problem, but may be inefficient for another. Therefore, we present a~benchmark tool which enables us to choose the best methods to generate an efficient solver in the end.

Thirdly, we run some experiments to show how the implemented improvements have enhanced the automatic generator. We compare the solvers generated by the new automatic generator and the solvers generated by the old implementation.

At last, we conclude by rewieving the contributions of this thesis.

\section{Notation used}
We have decided to use the notation from \cite{Cox-Little-Shea97} in the whole thesis. We just remind that polynomial is a sum of terms and term is a product of a coefficient and a monomial. Be aware that in some literature \cite{Becker93, F4, F5} the meanings of words term and monomial are switched.
