\chapter*{Abstract}
Many problems in computer vision lead to polynomial systems solving. Therefore, we need an easy way how to generate an efficient solver for each problem. On this purpose the automatic generator has been presented. In this thesis we improve the automatic generator so we will be able to generate more efficient and numerically stable solvers.

To improve the automatic generator we review and implement some methods used in the state of the art Gr\"obner basis solvers. Especially, we focus on the $F_4$ Algorithm by Jean-Charles Faug\`ere. Solvers, generated by the automatic generator, can be speed up when efficient methods are used to work with sparse matrices. We describe and implement method which is based on matrix partitioning. This method significantly speeds up the Gauss-Jordan elimination of sparse matrices.

We demonstrate the enhancements of the automatic generator on several important minimal problems. We show that the solvers generated by the new automatic generator are faster and numerically more stable than the solvers generated by the old version of the automatic generator.

\paragraph{Keywords:} computer vision, robotics, minimal problems, polynomials equations, Gr\"ob\-ner basis

\begin{otherlanguage}{czech}
\chapter*{Abstrakt}
Mnoho problémů v počítačovém vidění vede na řešení polynomiálních rovnic. Proto potřebujeme jednoduchý způsob, jak generovat efektivní postupy řešení každého z problémů. Z toho důvodu byl představen automatický generátor. V této práci vylepšíme automatický generátor, takže budeme schopni generovat ještě rychlejší a numericky stabilnější postupy řešení polynomiálních systémů.

Abychom mohli vylepšit automatický generátor, prozkoumáme a následně implementujeme některé metody používané v současných nástrojích na řešení soustav polynomiálních rovnic pomocí Gr\"obnerových bází. Zaměříme se zejména na algoritmus $F_4$ představený Jean-Charlesem Faug\`erem. Postupy řešení problémů, vygenerované pomocí automatického generátoru, mohou být ještě dále zrychleny, pokud použijeme efektivní metody na práci s řídkými maticemi. Popíšeme a implementujeme metodu, která je založená na rozkladu matic. Tato metoda výrazně urychluje Gauss-Jordanovu eliminaci řídkých matic.

Vylepšení automatického generátoru předvedeme na několika významných mini\-mál\-ních problémech. Ukážeme, že postupy řešení problémů vygenerované novým automatickým generátorem jsou rychlejší a numericky stabilnější než postupy vygenerované původní verzí automatického generátoru.

\paragraph{Klíčová slova:} počítačové vidění, robotika, minimální problémy, polynomiální rovnice, Gr\"obnerovy báze
\end{otherlanguage}
